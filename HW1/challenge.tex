\textbf{Claim}: Union operation cannot be expressed in terms of union, cartesian product, projection, and selection. \\
  \textbf{Observation}: By definition of union, the two arguments must have the same arity but need not have the same attributes. 
  When taking a union $R \cup S$, where $R$ and $S$ have the same arity of $k$, there are $2k$ attributes to consider. The only 
  way for $R \cup S$ to result in a new $k$-ary relation is to combine two attributes, which might be different, into one. 
  We need to show that the output of every relational algebra expression in the other four operations does not combine two separate 
  attributes into one. \\ 
  \textbf{Proof}: We perform a structural induction proof over the context-free grammar for relational algebra expressions:
  \begin{equation*}
    E := R, S | (E_1 - E_2) | (E_1 \times E_2) | \pi_{X}(E) | \sigma_{\theta}(E)
  \end{equation*}
  where $R, S$ are $k$-ary relations whose attributes are different, $X$ is a list of attributes, and $\theta$ is a condition. 
  By induction on the length of $E$, we assume that any expression $E'$ smaller than $E$ results in a new relation whose 
  values are associated with the same original attributes of the inputs. There are six cases to consider: 
  \begin{itemize}
    \item \textbf{Case 1 (Relation)}: $E = R$. It is trivial that all the attributes of $E$ contain the original values.
    \item \textbf{Case 1 (Relation)}: $E = S$. It is trivial that all the attributes of $E$ contain the original values.
    \item \textbf{Case 3 (Difference)}: $E = E_1 - E_2$ \\
    By the induction hypothesis, all the values in $E_1$ and $E_2$ are associated with the same original attributes of the 
    inputs. It follows that $E_1 - E_2$ only removes tuples in $E_1$ that are also in $E_2$. Hence, no two attributes are 
    combined, and all the values in $E$ remain the same attributes that they were in before the operation. 
     
    \item \textbf{Case 4 (Cartesian product)}: $E = E_1 \times E_2$ \\
    By the induction hypothesis, all the values in $E_1$ and $E_2$ are associated with the same original attributes of the 
    inputs. By the definition of cartesian product, we have:
    \begin{equation*}
      E_1 \times E_2 = \{(a_1,...,a_m,b_1,...,b_n) : (a_1,...,a_m) \in E_1 \land (b_1,...,b_n) \in E_2\}
    \end{equation*}
    where $m$ is the arity of $E_1$ and $n$ is the arity of $E_2$. It follows that $E_1 \times E_2$ results in a $(m+n)$-ary 
    relation, and no two attributes are merged. Hence, all the values in $E$ remain the same attributes that they were in 
    before the operation. 

    \item \textbf{Case 5 (Projection)}: $E = \pi_{i_1,...,i_m}(E_1)$ \\
    By the induction hypothesis, all the values in $E_1$ are associated with the same original attributes of the inputs. It 
    follows $\pi_{i_1,...,i_m}(E_1)$ only removes the attributes in $E_1$ which are not listed in the set $\{i_1,...,i_m\}$. 
    No two attributes are combined, and hence, all the values in $E$ remain in the same attributes that they were in before 
    the operation.

    \item \textbf{Case 6 (Selection)}: $E = \sigma_{\theta}(E_1)$ \\
    By the induction hypothesis, all the values in $E_1$ are associated with the same original attributes of the inputs. It 
    follows that $\sigma_{\theta}(E_1)$ only removes tuples that do not satisfy $\theta$. No two attributes are 
    combined, and hence, all the values in $E$ remain in the same attributes that they were in before the 
    operation. 
  \end{itemize}  