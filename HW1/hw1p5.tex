\begin{enumerate}
  \item \textbf{Claim}: Difference operation cannot be expressed in terms of union, cartesian product, projection, and selection. \\
  \textbf{Observation}: Difference operation is the only one whose output size can decrease if its input size increases. For instance, 
  if we take the difference $R - S$, where $R$ and $S$ have the same arity, and increase the size of $S$ by some $x$ number of tuples, 
  there are two cases:
  \begin{enumerate}
    \item $x$ does not add to $|R \cap S|$, then $|R - S|$ remains the same.
    \item $x$ adds to $|R \cap S|$, then $|R - S| = |R| - |R \cap S|$ decreases.
  \end{enumerate}
  We need to show that the output of every relational algebra expression in the other four operations does not decrease if its input 
  size increases. \\ 
  \textbf{Proof}: We perform a structural induction proof over the context-free grammar for relational algebra expressions:
  \begin{equation*}
    E := R, S | (E_1 \cup E_2) | (E_1 \times E_2) | \pi_{X}(E) | \sigma_{\theta}(E)
  \end{equation*}
  where $R, S$ are $k$-ary relations, $X$ is a list of attributes, and $\theta$ is a condition. By induction on the length of $E$, 
  we assume that any expression $E'$ smaller than $E$ does not decrease its output size if its input size increases. There are 
  six cases to consider: 
  \begin{itemize}
    \item \textbf{Case 1 (Relation)}: $E = R$. It is trivial that $|E|$ increases if $|R|$ increases.
    \item \textbf{Case 2 (Relation)}: $E = S$. It is trivial that $|E|$ increases if $|S|$ increases.
    \item \textbf{Case 3 (Union)}: $E = E_1 \cup E_2$ \\
    If we increase the input size of $E$, then by the induction hypothesis, $|E_1|$ and $|E_2|$ either stay the same or increase. Since 
    $|E_1 \cup E_2| = |E_1| + |E_2| - |E_1 \cap E_2|$. There are four cases to consider:
    \begin{enumerate} 
      \item $|E_1|$ and $|E_2|$ stay the same, then $|E_1 \cap E_2|$ stays the same and $|E_1 \cup E_2|$ does not change.
      \item $|E_1|$ increases by some $x > 0$. Then there are three possible consequences:
      \begin{enumerate}
        \item $|E_1 \cap E_2|$ remains the same. Then $|E_1| + x + |E_2| - |E_1 \cap E_2| > |E_1| + |E_2| - |E_1 \cap E_2|$, and 
        $|E_1 \cup E_2|$ increases by $x > 0$.
        \item $|E_1 \cap E_2|$ increases by $x$. Then $|E_1| + x + |E_2| - (|E_1 \cap E_2| + x) = |E_1| + |E_2| - |E_1 \cap E_2|$. 
        Therefore, $|E_1 \cup E_2|$ stays the same.
        \item $|E_1 \cap E_2|$ increases by some positive $m < x$. Then $|E_1| + x + |E_2| - (|E_1 \cap E_2| + m) = |E_1| + |E_2| -  
        |E_1 \cap E_2| + (x - m) > |E_1| + |E_2| - |E_1 \cap E_2|$. Then $|E_1 \cup E_2|$ increases by $(x - m)$.
      \end{enumerate}
      \item $|E_2|$ increases (same argument as (b)).
      \item $|E_1|$ increases by $x > 0$ and $|E_2|$ increases by $y > 0$. Again, we have three possible consequences:
      \begin{enumerate}
        \item $|E_1 \cap E_2|$ remains the same. Then $|E_1| + x + |E_2| + y - |E_1 \cap E_2| > |E_1| + |E_2| - |E_1 \cap E_2|$, and 
        $|E_1 \cup E_2|$ increases by $(x + y)$.
        \item $|E_1 \cap E_2|$ increases by $(x + y)$. Then $|E_1| + x + |E_2| + y - (|E_1 \cap E_2| + x + y) = |E_1| + |E_2| - 
        |E_1 \cap E_2|$. Therefore $|E_1 \cup E_2|$ stays the same.
        \item $|E_1 \cap E_2|$ increases by some positive $m < (x + y)$. Then $|E_1| + x + |E_2| + y - (|E_1 \cap E_2| + m) = |E_1|  
        + |E_2| - |E_1 \cap E_2| + (x + y - m) > |E_1| + |E_2| - |E_1 \cup E_2|$. Then $|E_1 \cup E_2|$ increases by $(x + y - m)$.
      \end{enumerate}
    \end{enumerate}
    In all cases, $|E|$ either stays the same or increases when its inputs increase.

    \item \textbf{Case 4 (Cartesian product)}: $E = E_1 \times E_2$ \\
    If we increase the input size of $E$, then by the induction hypothesis, $|E_1|$ and $|E_2|$ either stay the same or increases. Since 
    $|E_1 \times E_2| = |E_1| \times |E_2|$, $|E|$ either stays the same if both $|E_1|$ and $|E_2|$ do not change or increases if either 
    or both $|E_1|$ and $|E_2|$ increase.

    \item \textbf{Case 5 (Projection)}: $E = \pi_{i_1,...,i_m}(E_1)$ \\
    If we increase the input size of $E$, then by the induction hypothesis, $|E_1|$ either stays the same or increase. There are four 
    cases to consider:
    \begin{enumerate}
      \item $|E_1|$ either stays the same, then $|\pi_{i_1,...,i_m}(E_1)|$ stays the same.  
      \item $|E_1|$ increases with non-duplicate values in the projected attributes, then \\ $|\pi_{i_1,...,i_m}(E_1)|$ increases.
      \item $|E_1|$ increases with some duplicate values in the projected attributes, then \\ $|\pi_{i_1,...,i_m}(E_1)|$ still increases.
      \item $|E_1|$ increases with all duplicate values in the projected attributes, then \\ $|\pi_{i_1,...,i_m}(E_1)|$ does not change.
    \end{enumerate} 
    In all cases, $|E|$ either stays the same or increases when its inputs increase. 

    \item \textbf{Case 6 (Selection)}: $E = \sigma_{\theta}(E_1)$ \\
    If we increase the input size of $E$, then by the induction hypothesis, $|E_1|$ either stays the same or increases. If $|E_1|$ remains 
    the same, then $|\sigma_{\theta}(E_1)|$ does not change. On the other hand, if $|E_1|$ increases, then there are more candidates for 
    the tested condition $\theta$. It follows that $|E|$ either remains the same if all the new tuples fail the 
    condition $\theta$ or increases if some new tuples satisfy the condition. 
  \end{itemize}  

  \item \textbf{Claim}: If relational algebra expressions are built using union, intersection, cartesian product, 
  projection, and selection, then this version of relational algebra is not relationally complete. \\
  \textbf{Justification}: Let's call this new algebra $A$. By the definition of relational completeness, every expression $E$ in 
  relational algebra should have an equivalent in $A$. In other words, the objective is to find the equivalence of the five basic 
  operations (difference, union, cartesian product, projection, selection) in $A$. We want to represent the difference operation 
  in $A$. We showed in (1) that the difference operation cannot be expressed in terms of union, cartesant product, projection, and 
  selection. We only need to show that the intersection operation cannot express difference either. As we have argued in (1), one unique 
  property of difference is that its output size might decrease if its input size increases. By induction on the 
  length of $E$, we assume that any expression $E'$ smaller than $E$ does not decrease its output size if its input size increases. 
  We need to show that all the expressions $E = E_1 \cap E_2$ do not decrease their output size if their input size increases. 
  By the induction hypothesis, $|E_1|$ and $|E_2|$ either stay the same or increase if we increase the input size of $E$. There are four 
  cases to consider: 
  \begin{enumerate}
    \item $|E_1|$ and $|E_2|$ stay the same, then $|E_1 \cap E_2|$ stays the same.
    \item $|E_1|$ increases by some $x > 0$: 
    \begin{enumerate}
      \item Some of $x$ tuples are common tuples between $E_1$ and $E_2$, then $|E_1 \cap E_2|$ increases. 
      \item None of $x$ tuples are common tuples between $E_1$ and $E_2$, then $|E_1 \cap E_2|$ stays the same.
    \end{enumerate}
    \item $|E_2|$ increases (same argument as (b)).
    \item $|E_1|$ increases by $x > 0$ and $|E_2|$ increases by $y > 0$:
    \begin{enumerate}
      \item Some of $(x + y)$ tuples are common tuples between $E_1$ and $E_2$, then $|E_1 \cap E_2|$ increases. 
      \item None of $(x + y)$ tuples are common tuples between $E_1$ and $E_2$, then $|E_1 \cap E_2|$ stays the same.
    \end{enumerate}
  \end{enumerate}
  Hence, in no way does the intersection operation decrease its output size when its inputs increase their size. 
  In other words, intersection cannot express the difference operation. Since there does not exist any operation in $A$ that 
  could express the difference operation, this new version of algebra is not relationally complete. 

  \item \textbf{Claim}: Quotient operation cannot be expressed in terms of union, cartesian product, projection, and selection 
  alone. \\
  \textbf{Proof}: Without loss of generality, let's consider the quotient operation $R \div S$ on a $5$-ary relation $R$ and a binary 
  relation $S$. By the definition of quotient, $R \div S$ only includes $(a_1,a_2,a_3) \in \pi_{1,2,3}(R)$ if $\forall (b_1,b_2) \in S$, 
  $(a_1,a_2,a_3,b_1,b_2) \in R$. In other words, we need to find all the tuples $(a_1,a_2,a_3)$ in $R$ that are paired up with 
  every tuple in $S$. Suppose $|S| = m$, and there are $n$ tuples in $R$ that satisfy the condition for $R \div S$. Let's consider a new 
  $S^*$ by adding a new tuple $(b_1^*,b_2^*)$ to $S$. It follows that we only need to consider among the $n$ tuples to see which ones are also 
  matched up with $(b_1^*,b_2^*)$. We do not have to consider the other tuples in $R$ because they already failed to get matched up with 
  the original $m$ tuples in $S$. In other words, $R - S^*$ can only output at most $n$ tuples resulted from $R - S$. Certainly, there 
  might be tuples $(a_1,a_2,a_3)$ among these $n$ candidates where $(a_1,a_2,a_3,b_1^*,b_2^*) \notin R$ and so are not included 
  in $R - S^*$. Hence, $|R - S^*| \leq n = |R - S|$. In conclusion, the output of the quotient operation may decrease if its input size 
  increases. We have showed in (1) that the output of every relational algebra expression using union, cartesian product, projection, 
  and selection does not decrease if its input size increases. Therefore, none of these operations and their composition can express 
  the quotient operation.
\end{enumerate}