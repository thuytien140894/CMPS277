Given a Boolean conjunctive query $q$
\begin{equation*}
  q :- q_1,q_2,...,q_n.
\end{equation*}
\begin{enumerate}
  \item If both $q_1$ and $q_2$ are minimal equivalent conjunctive queries of $q$, then the 
  canonical database of $q_1$ is isomorphic to the canonical database of $q_2$. \\ \\
  \textbf{Proof:} Let $q_1$ and $q_2$ be the two minimal equivalent conjunctive queries to $q$. 
  It follows that $q_1 \equiv q_2$. Hence, we have $q_1 \subseteq q_2$ and $q_2 \subseteq q_1$. 
  By the Homomorphism Theorem, there is a homomorphism $h_1$: $I^{q_1} \rightarrow I^{q_2}$, 
  and there is also a homomorphism $h_2$: $I^{q_2} \rightarrow I^{q_1}$. We have 
  \begin{equation*}
    h_1 \circ h_2 = I^{q_2} \rightarrow I^{q_2} = id_{I^{q_2}} 
  \end{equation*}
  Hence, $h_2$ is the inverse of $h_1$, and $h_1$ is bijective. Since there exists an 
  isomorphism $h_1$: $I^{q_1} \rightarrow I^{q_2}$, $q_1$ and $q_2$ are isomorphic.

  \item There is a Boolean conjunctive query $q'$ that is a minimal equivalent query to $q$ 
  and is obtained from $q$ by removing zero or more conjuncts of $q$. \\ \\
  \textbf{Proof:} Any conjunctive query has an equivalent query of the same 
  number of atoms by renaming its variables. Therefore, the size of a minimal equivalent 
  conjunctive query should be at most the size of $q$. We can obtain a minimal 
  equivalent conjunctive query in a greedy way. We iterate through all atoms of $q$ and 
  see if we can obtain another equivalent query by removing one atom. If another 
  equivalent query is found, we repeat the same process for the remainting atoms. 
  Otherwise, we have found the minimal one.   
\end{enumerate}