Let $q_1$ and $q_2$ be the Boolean conjunctive queries:
\begin{equation*}
  q_1 :- E(x_1,x_2),E(x_2,x_1),E(x_2,x_3),E(x_3,x_2),E(x_3,x_4),E(x_4,x_3),E(x_4,x_1),E(x_1,x_4).
\end{equation*}
and 
\begin{align*}
  q_2 :- E(x_1,x_2),E(x_2,x_1),E(x_2,x_3),E(x_3,x_2),E(x_3,x_4), \\ 
  E(x_4,x_3),E(x_4,x_1),E(x_1,x_4),E(x_1,x_3),E(x_3,x_1).
\end{align*} 
\textbf{Proof:} The canonical instances of $q_1$ and $q_2$ are given as 
\begin{align*}
  I^{q_1} = &E(x_1,x_2),E(x_2,x_1),E(x_2,x_3),E(x_3,x_2), \\
            &E(x_3,x_4),E(x_4,x_3),E(x_4,x_1),E(x_1,x_4) \\
  I^{q_2} = &E(x_1,x_2),E(x_2,x_1),E(x_2,x_3),E(x_3,x_2),E(x_3,x_4), \\ 
            &E(x_4,x_3),E(x_4,x_1),E(x_1,x_4),E(x_1,x_3),E(x_3,x_1)
\end{align*}
\begin{enumerate}
    \item $q_1 \not \subseteq q_2$  \\
    If we use a function $h$ where
    \begin{shiftedflalign*}
      & h(x_1) = x_1 & \\
      & h(x_2) = x_2 & \\
      & h(x_3) = x_3 & \\
      & h(x_4) = x_4 & 
    \end{shiftedflalign*}
    to map most tuples from $I^{q_2}$ to $I^{q_1}$, then we cannot map $(x_1,x_3)$ and $(x_3,x_1)$ 
    from $I^{q_2}$ to $I^{q_1}$. Hence, there does not exist any homomorphism $h$: $I^{q_2} 
    \rightarrow I^{q_1}$. Hence it follows that $q_1 \not \subseteq q_2$.

    \item $q_2 \subseteq q_1$ \\
    There is homomorphism $h$: $I^{q_1} \rightarrow I^{q_2}$ where
    \begin{shiftedflalign*}
      & h(x_1) = x_1 & \\
      & h(x_2) = x_2 & \\
      & h(x_3) = x_3 & \\
      & h(x_4) = x_4 & 
    \end{shiftedflalign*}
    Hence, $q_2 \subseteq q_1$ by the Homomorphism Theorem.
    
    \item A conjunctive query $q_3$ such that $q_3$ is a minimal conjunctive query equivalent to $q_1$
    \begin{equation*}
      q_3 :- E(x_1,x_2),E(x_2,x_1).
    \end{equation*} 
    \begin{enumerate}
      \item We need to show that $q_1 \equiv q_3$. The canonical instance of $q_3$ is $I^{q_3} = E(x_1,x_2),E(x_2,x_1)$. 
      There is a homomorphism $h_1: I^{q_1} \rightarrow I^{q_3}$ where 
      \begin{shiftedflalign*}
        & h_1(x_1) = x_1 & \\
        & h_1(x_2) = x_2 & \\
        & h_1(x_3) = x_1 & \\
        & h_1(x_4) = x_2 & 
      \end{shiftedflalign*}
      Hence, $q_3 \subseteq q_1$ by the Homomorphism Theorem (1). There is also a homomorphism $h_2$: $I^{q_3} 
      \rightarrow I^{q_1}$ where 
      \begin{shiftedflalign*}
        & h_2(x_1) = x_1 & \\
        & h_2(x_2) = x_2 &  
      \end{shiftedflalign*}
      Therefore, $q_1 \subseteq q_3$ by the Homomorphism Theorem (2). By (1) and (2), $q_1 \equiv q_3$.

      \item To show that $q_3$ is minimal, we consider all other conjunctive queries that are equivalent to $q_1$ 
      \begin{align*}
        &q_3' :- E(x_1,x_2),E(x_2,x_1),E(x_2,x_3),E(x_3,x_2). \\ 
        &q_3'' :- E(x_1,x_2),E(x_2,x_1),E(x_2,x_3),E(x_3,x_2),E(x_3,x_4),E(x_4,x_3).
      \end{align*}
      $q_3$ has the fewest conjuncts compared to all the conjunctive queries equivalent to $q_1$.
    \end{enumerate}

    \item A conjunctive query $q_4$ such that $q_4$ is a minimal conjunctive query equivalent to $q_2$
    \begin{equation*}
      q_4 :- E(x_1,x_2),E(x_2,x_1),E(x_2,x_3),E(x_3,x_2),E(x_1,x_3),E(x_3,x_1). 
    \end{equation*}
    \begin{enumerate}
      \item We need to show that $q_2 \equiv q_4$. The canonical instance of $q_4$ is $I^{q_4} = E(x_1,x_2),E(x_2,x_1),\\
      E(x_2,x_3),E(x_3,x_2),E(x_1,x_3),E(x_3,x_1)$. There is a homomorphism $h_1: I^{q_2} \rightarrow I^{q_4}$ where
      \begin{shiftedflalign*}
        & h_1(x_1) = x_1 & \\
        & h_1(x_2) = x_2 & \\
        & h_1(x_3) = x_3 & \\
        & h_1(x_4) = x_2 & 
      \end{shiftedflalign*} 
      By the Homomorphism Theorem, $q_4 \subseteq q_2$. There is also a homomorphism $h_2$: $I^{q_4} \rightarrow 
      I^{q_2}$ where
      \begin{shiftedflalign*}
        & h_2(x_1) = x_1 & \\
        & h_2(x_2) = x_2 & \\
        & h_2(x_3) = x_3 &
      \end{shiftedflalign*} 
      Hence, $q_2 \subseteq q_4$ by the Homomorphism Theorem.

      \item To show that $q_4$ is minimal, we consider all other conjunctive queries that are equivalent to $q_2$ 
      \begin{align*}
        &q_4' :- E(x_1,x_2),E(x_2,x_1),E(x_2,x_3),E(x_3,x_2),E(x_3,x_4),E(x_4,x_3),E(x_1,x_3),E(x_3,x_1). 
      \end{align*}
      $q_4$ has the fewest conjuncts compared to all the conjunctive queries equivalent to $q_2$.
    \end{enumerate}
\end{enumerate}